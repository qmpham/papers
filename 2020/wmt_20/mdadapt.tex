%
% File emnlp2020.tex
%
%% Based on the style files for ACL 2020, which were
%% Based on the style files for ACL 2018, NAACL 2018/19, which were
%% Based on the style files for ACL-2015, with some improvements
%%  taken from the NAACL-2016 style
%% Based on the style files for ACL-2014, which were, in turn,
%% based on ACL-2013, ACL-2012, ACL-2011, ACL-2010, ACL-IJCNLP-2009,
%% EACL-2009, IJCNLP-2008...
%% Based on the style files for EACL 2006 by 
%%e.agirre@ehu.es or Sergi.Balari@uab.es
%% and that of ACL 08 by Joakim Nivre and Noah Smith

\documentclass[11pt,a4paper]{article}
\usepackage[hyperref]{emnlp2020}
\usepackage{times}
\usepackage{latexsym}
\renewcommand{\UrlFont}{\ttfamily\small}

\usepackage{graphicx}
\usepackage{xcolor}
\usepackage{longtable}
\usepackage{tikz}
\usetikzlibrary{calc}
\usepackage[draft]{todo}
\usepackage{xspace}
\usepackage{amsmath,amsfonts,amssymb}
\usepackage{mathrsfs}
\newcommand{\R}{\ensuremath{\mathbb{R}}}

% This is not strictly necessary, and may be commented out,
% but it will improve the layout of the manuscript,
% and will typically save some space.Boldface denotes significant gains with respect to \system{Mix-Nat} (or \system{Mix-Nat-RNN}, for WDCMT), underline denotes significant losses.
\usepackage{microtype}

%\aclfinalcopy % Uncomment this line for the final submission
%\def\aclpaperid{***} %  Enter the acl Paper ID here

%\setlength\titlebox{5cm}
% You can expand the titlebox if you need extra space
% to show all the authors. Please do not make the titlebox
% smaller than 5cm (the original size); we will check this
% in the camera-ready version and ask you to change it back.

\newcommand{\fyTodo}[1]{\Todo[FY:]{\textcolor{orange}{#1}}}
\newcommand{\fyTodostar}[1]{\Todo*[FY:]{\textcolor{orange}{#1}}}
\newcommand{\fyDone}[1]{\done[FY]\Todo[FY:]{\textcolor{orange}{#1}}}
\newcommand{\fyFuture}[1]{\done[FY]\Todo[FY:]{\textcolor{red}{#1}}}
\newcommand{\fyDonestar}[1]{\done[FY]\Todo[FY:]{\textcolor{orange}{#1}}}
\newcommand{\jcTodo}[1]{\Todo[JC:]{\textcolor{red}{#1}}}
\newcommand{\jcDone}[1]{\done[JC]\Todo[JC:]{\textcolor{red}{#1}}}

\newcommand{\mpTodo}[1]{\Todo[MP:]{\textcolor{green}{#1}}}
\newcommand{\mpDone}[1]{\done[MP]\Todo[MP:]{\textcolor{green}{#1}}}
\usepackage{mathtools,xparse}
\DeclarePairedDelimiter{\abs}{\lvert}{\rvert}
\DeclarePairedDelimiter{\norm}{\lVert}{\rVert}
\NewDocumentCommand{\normL}{ s O{} m }{%
  \IfBooleanTF{#1}{\norm*{#3}}{\norm[#2]{#3}}_{2}%
}
\newcommand{\src}{\ensuremath{\mathbf{f}}} % source sentence
\newcommand{\trg}{\ensuremath{\mathbf{e}}} % target sentence
\newcommand{\domain}[1]{\texttt{\textsc{#1}}}
\newcommand{\system}[1]{\texttt{\textbf{#1}}}
\newcommand{\vlambda}{\ensuremath{\boldsymbol\lambda}\xspace} % parameters vector for a distribution
\newcommand{\indic}[1]{\ensuremath{\mathbb{I}(#1)}}
% \newcommand{\SB}[1]{\textcolor{green}{#1}}
% \newcommand{\SW}[1]{\textcolor{red}{#1}}
\newcommand{\SB}[1]{\textbf{#1}}
\newcommand{\SW}[1]{\underline{#1}}
\renewcommand\textfraction{.1}
\renewcommand\floatpagefraction{.8}

\newcommand\BibTeX{B\textsc{ib}\TeX}

\iffalse   % removes draft highlighting for final version
\renewcommand{\draftAdd}[2]{#1}
\fi

% -- \title{Ablation study on the residual adapter in Neural Machine Translation}
\title{A Study of Residual Adapters for Multi-Domain Neural Machine Translation}

\author{First Author \\
  Affiliation / Address line 1 \\
  Affiliation / Address line 2 \\
  Affiliation / Address line 3 \\
  \texttt{email@domain} \\\And
  Second Author \\
  Affiliation / Address line 1 \\
  Affiliation / Address line 2 \\
  Affiliation / Address line 3 \\
  \texttt{email@domain} \\}

\date{}

\begin{document}
\maketitle
\begin{abstract}
% Among the approaches for adapting a pretrained NMT model to a specific domain, finetuning is the dominant method. Finetuning proposes 2 approaches: simply continuing training all the parameters \cite{Luong15stanford}; training addition adapters while freezing the pretrained model\cite{Bapna19simple,Vilar18learning}. The second approach has several advantages including preserving the pretrained model and the legerity of the adapters. However, the behavior of these adapters has not been further studied. The objective of this paper is to give an ablation study on the use of residual adapter in the domain adaptation problem.
% \mpTodo{correcting abstract}
% **** 
\fyDone{Citation-free abstract}
Domain adaptation is an old and vexing problem for machine translation systems. The most common and successful approach to supervised adaptation is to fine-tune a baseline system with in-domain parallel data. Standard fine-tuning however modifies all the network parameters, which makes this approach computationally costly and prone to overfitting. A recent, lightweight approach, instead augments a baseline model with supplementary (small) adapter layers, keeping the rest of the model unchanged. This has the additional merit to leave the baseline model intact and adaptable to multiple domains. In this paper, we conduct a thorough analysis of the adapter model in the context of a multidomain machine translation task. We contrast multiple implementations of this idea using two language pairs. Our main conclusions are that residual adapters provide a fast and cheap method for supervised multi-domain adaptation; our two variants prove as effective as the original adapter model and open perspective to also make adapted models more robust to label domain errors.
\fyDone{abstract to be continued}

\end{abstract}
\section{Introduction \label{sec:intro}}
\mpDone{write introduction} \fyDone{Citations in chronological order}\fyDone{Split long sentences}
Owing to multiple improvements, Neural Machine Translation (NMT) \cite{Kalchbrenner13recurrent,Sutskever14sequence,Bahdanau15learning,Vaswani17attention} nowadays delivers useful outputs for many language pairs. However, as many deep learning models, NMT systems need to be trained with sufficiently large amounts of data to reach their best performance. Therefore, the quality of translation of NMT models is still limited in low-resource language or domain conditions \cite{duh13adaptation,zoph16transfer,koehn17six}. While many approaches have been proposed to improve the quality of NMT models in low-resource domains (see the recent survey of \citet{Chu18asurvey}), full fine-tuning \cite{Luong15stanford,neubig18rapid} of a generic baseline model remains the dominant supervised approach when adapting NMT models to specific domains.

Under this view, building adapted systems is a two-step process: (a) one first trains NMT with the largest possible parallel corpora, possibility aggregating texts from multiple, heterogeneous sources; (b) assuming that in-domain parallel documents are available for the domain of interest, one then adapts the pre-trained model by resuming training with the sole in-domain corpus. It is a conjecture that pre-trained model constitutes a better initialization than a random one, especially when adaptation data is scarce. Indeed, studies of transfer learning for NMT such as \cite{artetxe20cross,aji20neural} have confirmed this claim in extensive experiments. Full fine-tuning, that adapts all the parameters of a baseline model usually significantly improves the quality of the NMT for the chosen domain. However, it also yields large losses in translation quality for other domains, a phenomenon referred to as ``catastrophic forgetting'' in the neural network literature \cite{McCloskey89catastrophic}. Therefore, a fully fine-tuned model is \emph{only useful to one target domain}. As the number of domains to handle grows, training and maintaining a separate model for each task can quickly become tedious and resource-expensive .\fyDone{Fix this sentence.}

\cite{Vilar18learning,Bapna19simple} have recently proposed a simple and lightweight domain adaptation method that also preserves the value of pre-trained models, based on small adapters component that are plugged in each hidden layer. These adapters are trained only with the in-domain data, keeping the pre-trained model frozen. Because these additional adapters are very small compared to the size of the baseline model, their use significantly reduces the cost of training and maintaining fine-tuned models, while still delivering a performance that is close to that of full fine-tuning.

In this paper, we would like to extend this architecture to improve NMT in several settings that still challenge automatic translation, such as translating texts from multiple topics, genre or domains, in the face of unbalanced data distributions. Furthermore, as the notion of ``domains''  is not always well established, another practical setting is the translation of texts mixing several topics / domains. An additional requirement is to translate texts from domains unseen in training, based only on the unadapted system, which should then be made as strong as possible. 
% Residual adapters allow us to adapt NMT model to any specific domain in a computationally cheap way. Could we adapt NMT model to noisy text and topical text at once? While residual adapters are good at adapting to one specific domain, their performance still dramatically decrease for the other domains seen in training, as we show in our experiments. Therefore, we have to decide whether to use residual adapters manually, i.e, we have to know to which domain the text belongs a priori. Therefore, we would like to fuse a domain classifier to the architecture in order to weight the contribution of the adapters with respect to the topic-relatedness of the text.
\fyDone{Say differently: various implementations}

In this context, our main contribution is a thorough experimental study of the use of residual adapters for multi-domain translation. We notably explore ways to adjust and/or regularize adapter modules to handle situations where the adaptation data is very small. We also propose and contrast two new variants of the residual architecture: in the first one (\emph{highway residual adapters}), adaptation still affects each layer of the architecture, but its effect is delayed till the last layer, thus making the architecture more modular and adaptive; in the second one (\emph{gated residual adapters}), we explore ways to improve performance in the face of train-test data mismatch. We experiment with two language pairs and report results that illustrate the flexibility and effectiveness of these architectures. 
\fyDone{One bit of a conclusion here}\fyFuture{Build a proper training scenario for these two conditions}

\section{Residual adapters \label{sec:res}}
In this section, we describe the basic version of the residual adapter architectures, as well as two original variants of this model.

\subsection{Basic architecture \label{ssec:architecture}}
\fyDone{More contexts and notations from the transformer}\fyDone{Encoder / decoder layers}

\subsubsection{The computation of adapter layers}
Our reference architecture is the Transformer model of \cite{Vaswani17attention}, which we assume contains a stack of layers both on the encoder and the decoder sides. Each layer contains two subparts, an attention layer and a dense layer. Details vary from one implementation to another, we simply contend here that each layer $i \in \{1 \dots L\}$ (in the encoder or the decoder) computes a transform of a fixed-length sequence of $d$-dimensional input vectors $h^{i}$ into a sequence of output vectors $h^{i+1}$.

The residual adapter architecture extends this model by adding one new computation module for each layer. The $i^{\text{th}}$ residual adapter thus modifies $h^i$ by computing the following transformations:\fyDone{Use align env}
\begin{align*}
  h^{i}_1 &= \mathbf{W}_{db}^{i}h^{i} + b^i_{1} \\
  h^{i}_2 &= \mathbf{ReLU}(h_1^{i})
  \\
%\end{align*}
%\begin{align*}
  h^{i}_3 &= \mathbf{W}_{bd}^{i}h_2^{i} + b^i_{2} \\
  \bar{h}^{i} &= h^{i}_3 + h^i.
\end{align*}
Overall, the  $i^{\text{th}}$ adapter is thus parameterized by matrices $\displaystyle{\mathbf{W}_{db}^{i}\in\mathbb{R}^{d\times b}}$,$\displaystyle{\mathbf{W}_{bd}^{i}\in\mathbb{R}^{b\times d}}$, bias vectors $\displaystyle{b^i_{1} \in \mathbb{R}^{b}}$, $\displaystyle{b^i_{2} \in \mathbb{R}^{d}}$, with $b$ the dimension of the adapter ($d \gg b$)\fyDone{Check this}. For the sake of brevity, we will simply denote $h^{i}_3 = \operatorname{ADAP}^{(i)}(h^i)$, and $\theta_{\operatorname{ADAP}^{(i)}}$ the corresponding set of parameters.\fyDone{or is it $h_i$ ?}\fyDone{attention aux matrices $W_i$}

The "adapted" hidden vectors $\bar{h}^i_{ 1\leq i \leq L-1}$, where $L$ is the number of layers, will then be the input of the $(i+1)^{\text{th}}$\fyDone{Self attention ?} layer; $\bar{h}^L$ is passed to the decoder if it belongs to the encoder side, or is the input of output layer if it belongs to the decoder side. Note that zeroing out all adapters enables us to recover the basic Transformer, with $\bar{h}^{i} = h^i$ for all $i$.

In the experiments of Section~\ref{sec:exp}, we use $2\times{}L=12$ residual adapters, one for each of the $L=6$ attention layers of the encoder and similarly for the decoder.\footnote{In the decoder, the stack of self-attention layers and cross encoder-decoder attention only counts as one attention layer and only serves one residual adapter.}

\subsubsection{Design space and variants \label{sssec:design-space}}
This general architecture leaves open many design choices pertaining to the details of the network organization, the training procedure and the corresponding objective function.

A first question concerns the number of adapter layers. While in principle, all Transformer layers can be subject to adaptation, it is nonetheless worthwhile to consider simpler adaptation schemes, which would only alter a limited number of layers. Such strategy might be especially relevant when the training data contains very small domains, as in the experiments of Section~\ref{sec:exp}, and for which a complete adaptation may not be necessary or/and or prone to overfitting. Likewise, it might be meaningful to explore ways to share subsets of adapters across domains. This in turn raises the issue of which layer(s) to adapt, a question that can be approached in the light of recent analyses of Transformers models, which conjecture that the higher layers encode global patterns with a more ``semantic'' interpretation, while the lower layers encode local patterns akin to morpho-syntactic information \cite{raganato18analysis}.

% The size of available corpora for each domain is extremely varying. For example, the corpus of domain \domain{tourism} in En-De contains only around $7\times10^3$ sentence pairs while the corpus of domain \domain{news} in En-De contains approximately $3\times10^6$ sentence pairs. For extremely small domains, we could economize the number of parameters by reducing the number of residual adapters in the architecture. While only a limited number of adapters are incorporated to the NMT model, which positions in the model are more important in the domain adaptation task? We would like to analyze the impact of position and number of residual adapters involved in the adapted model. It is conjectured that the higher layers represent more global patterns such as semantic while the lower layers represent more local patterns such as syntactic \cite{raganato18analysis}\fyTodo{Missing reference here}. The domain shift in local patterns and global patterns has not yet studied. In this paper, we do not intend to study this aspect of the domain adaptation problem. We would like to give an extensive comparison between domain adaptation in different levels in NMT model. 

A related question concerns the regularization of adapter layers to mitigate overfitting. Reducing the number of adapters, or their dimensions, is simple, but such choices are difficult to optimize numerically -- an issue that becomes important as the number of domain grows. Less naïve alternatives can also be entertained, such as applying weight decay or layer regularization to the adapter. Implementing these requires to modify the objective function in a way that still allows for a smooth optimization problem. For instance, weight decay applies a penalization on the weights of the adapters, complementing the cross-entropy term with a function of the norm of the parameters: \fyDone{The second summation also runs over $x,y$ ? I think not}
\begin{equation*}
  \begin{split}
    \bar{L} & = \frac{1}{\#(x,y)}\mathop{\sum}_{x,y}( - \log(P(y|x))) \\
    & + \lambda  \sum_{i \in \{1,..,6\} \otimes \{enc, dec\}} \normL{\theta_{\operatorname{ADAP}^{(i)}}}
  \end{split}
\end{equation*}
An alternative scheme is \emph{layer regularization}, which penalizes the output of the adapters, corresponding to the following objective:
\begin{equation*}
  \begin{split}
    \bar{L} & = \frac{1}{\#(x,y)}\mathop{\sum}_{x,y} (-\log(P(y|x)) \\
    & + \lambda \sum_{i \in \{1,..,6\} \otimes \{enc, dec\}} \normL{\operatorname{ADAP}^{(i)}(h_i(x,y))})
  \end{split}
\end{equation*}

Finally, another independant design choice relates to the training strategy for adapters. A first option is to generalize supervised domain adaptation to multi-domain adaptation and to proceed in two steps: (a) train a generic model with all the available data; (b) train each adapter layer with domain-specific data, keeping the generic model parameters unchanged. Another strategy is to adopt the view of \citet{Dredze08online}, where the multi-domain setting is viewed as an instance of multi-task learning \cite{Caruana97multitask} with each domain corresponding to a specific task. This suggest to train all the parameters from scratch, as we would do in a multi-task mode. The generic parameters will still depend on all the available data, while each adapter will only be trained with the corresponding in-domain data.\fyFuture{Does everyone need a domain? Do we need this discussion - talk about sentence level adaptation, better classifier pour MDL}

\subsection{Highway Residual Adapters \label{ssec:highway}}

In the basic architecture described in Section~\ref{ssec:architecture}, the computation performed by lower level layers will impact all the subsequent layers. In this section, we introduce an alternative implementation of the same idea, which however delays the adaptation of each layer to the last layer (of the encoder or the decoder) as depicted on Figure~\ref{fig:hrl-architecture}. While the basic architecture performs adaptation in sequence, we propose here to perform it in parallel. In this version, only the last hidden vector of the encoder (decoder) is thus modified according to:
\begin{equation}
  \bar{h}^L = h^L + \displaystyle{\mathop{\sum}_{1 \leq i \leq L} ADAP^i(h^i)} \label{eq:highway-output}
\end{equation}

One obvious benefit of this variant is that it allows us to reuse the hidden vectors $h^i$ of all hidden layers when computing an adapted output for several domains during the inference. In this situation, the forward step needs only to compute the hidden vectors $h^i$ once for the inner encoder layers, before an adapted sequence of vectors is computed at the topmost layer. Therefore, we can finetune the model to multiple domains at once without recomputing $h^i$. This variant also opens the way to more parameter sharing across adapters, a perspective that we will not explore further in this work. Instead, we use it to develop a second variation of the adapter model, that is presented in the next section.

\begin{figure}[htbp]
  \centering
  \includegraphics[scale=0.3]{fig/highway_residual}
  \caption{Highway residual adapter network}
  \label{fig:hrl-architecture}
\end{figure}

\subsection{Gated Residual Adapters \label{ssec:gate}}
\mpDone{Formalizing problem, network design, training algorithm}
The basic architecture presented above rests on a rather simplistic view of ``domains'' as made of well-separated and unrelated pieces of texts that are processed independently during adaptation. Likewise, when translating test documents, one needs to choose between either using one specific domain-adapted model or resorting to the generic model. In this context, using wrong domain labels can have a strong (negative) effect on translation performance. 

Therefore, we would like to design a version of residual adapters that is more robust to such domain errors. This variant, called the \emph{gated residual adapter model}, relies on the training of a supplementary trained component that will help decide whether to activate, on a word per word basis, a given residual layer, and to regulate the strength of this activation. To this end, we extend the highway version of residual adapters as follows.
\fyDone{Consistency of notations wrt section 2.1}

Formally, we replace the adapter computation of equation~\eqref{eq:highway-output} and take the adapted hidden (topmost) layer to be computed as (this is for domain $k$):
\begin{equation}
  \bar{h}^L = h^L + \displaystyle{\mathop{\sum}_{1 \leq i \leq L} \operatorname{ADAP}_k^i(h^i) \odot{} z_k(h^L)}, \label{eq:gated-output}
\end{equation}
where $z_k(h^L[t]) \in [0,1]$ measures the relatedness of the $t^{\text{th}}$ word $w_t$ to domain $k$. The more likely $w_t$ is in domain $k$, the larger $z_k(h^L[t])$ should be; conversely, for words\footnote{The term ``word'' is employed here by mere convenience, as systems only manipulate sub-lexical BPE units; furthermore, the values of the hidden representations $h^{i}$ at position $t$ depend upon all the other positions in the sentence.} that are not typical of any domain $k$ (eg.\ function words),  adaptation is minimum and the corresponding adapted encoder output ($\bar{h}^L[t]$) will remain close to the output of the generic model ($h^L[t]$). In our implementation, we incorporate two domain classifiers on top of the encoder and the decoder, that take the last hidden layer of the encoder (resp.\ decoder) as input and use the posterior probability $P(k|h^L[t])$ of domain $k$ as the value for $z_k(h^L[t])$.

Training gated residual adapters thus comprises three steps, instead of two for the baseline version:
\begin{enumerate}
\item learn a generic model with mixed corpora from multiple domains.
\item train a domain classifier on top of the encoder and decoder; during this step, the parameters of the generic model are frozen. This model computes the posterior domain probability $P(k|h^L[t])$ for each word $w_t$, based on the representation computed by the last layer.
\item train the parameters of adapters with in-domain data separately for each domain, while freezing all the other parameters.
\end{enumerate}
\fyFuture{is this classifier important, can we train with the rest of the system ? Yes we could, but do we want the encoder to be good at classification?}\fyFuture{Check writing to indicate label smoothing for this has been used.}\fyFuture{Can we train a more complex training model ? should we us emore monolingual data ? etc} 

\section{Experimental settings \label{sec:exp}}

\subsection{Data and metrics \label{ssec:corpora}}
We perform our experiments with two translation pairs involving multiple domains: English-French (En$\rightarrow$Fr) and English-German (En$\rightarrow$De). For the former pair, we use texts\footnote{Most corpora are available from the Opus web site: \url{http://opus.nlpl.eu}} initially from 6~domains, corresponding to the following data sources: the UFAL Medical corpus V1.0 (\domain{med})\footnote{\url{https://ufal.mff.cuni.cz/ufal_medical_corpus}}, the European Central Bank corpus (\domain{bank}) \cite{Tiedemann12parallel}; The JRC-Acquis Communautaire corpus (\domain{law}) \cite{Steinberger06acquis}, documentations for KDE, Ubuntu, GNOME and PHP from Opus collection \cite{Tiedemann09news}, collectively merged in a \domain{it}-domain, Ted Talks (\domain{talk}) \cite{Cettolo12wit}, and the Koran (\domain{rel}). Complementary experiments also use v12 of the News Commentary corpus (\domain{news}). Corpus statistics are in Table~\ref{tab:Corpora-en-fr}.  

\begin{table*}[htbp]
  \centering
  \begin{tabular}{ |lllllll|} %*{4}{|r|}}
    \hline
    %\multicolumn{4}{|l|}{Vocab size - En: 30,165, Fr: 30,398}\\
    \domain{med} & \domain{law} & \domain{bank} & \domain{it} & \domain{talk} & \domain{rel} & \domain{news} \\
    \hline
    2609 (0.68) & 190 (0.05)  & 501 (0.13) & 270 (0.07) & 160 (0.04) & 130 (0.03) & 260 (0) \\
    \hline
  \end{tabular}
\caption{Corpora statistics for En$\rightarrow$Fr : number of parallel lines ($\times 10^3$) and proportion in the basic domain mixture (which does not include the \domain{news} domain). \domain{med} is the largest domain, containing almost 70\% of the sentences, while \domain{rel} is the smallest, with only 3\% of the data.}
\label{tab:Corpora-en-fr}
\end{table*}

En$\rightarrow$De is a much larger task, for which we use corpora distributed for the News task of WMT20\footnote{\url{http://www.statmt.org/wmt20/news.html}} including: European Central Bank corpus (\domain{bank}),  European Economic and Social Committee corpus (\domain{eco}), European Medicines Agency corpus (\domain{med})\footnote{\url{https://tilde-model.s3-eu-west-1.amazonaws.com/Tilde_MODEL_Corpus.html}}, Press Release Database of European Commission corpus, News Commentary v15 corpus, Common Crawl corpus (\domain{news}), Europarl v10 (\domain{gov}), Tilde MODEL - czechtourism (\domain{tour})\footnote{\url{https://tilde-model.s3-eu-west-1.amazonaws.com/Tilde_MODEL_Corpus.html}}, Paracrawl and Wikipedia Matrix (\domain{web}). Statistics are in Table~\ref{tab:Corpora-en-de}.
\begin{table*}[htbp]
  \centering
  \begin{tabular}{ |lllllll|} %*{4}{|r|}}
    \hline
    %\multicolumn{4}{|l|}{Vocab size - En: 30,165, Fr: 30,398}\\
    \domain{bank} & \domain{eco} & \domain{med} & \domain{gov} & \domain{news} & \domain{tour} & \domain{web} \\
    \hline
    4 (0.00022) & 2857 (0.15) & 347 (0.018) & 1828 (0.095) & 3696 (0.19) & 7 (0.00039) & 10473 (0.54) \\
    \hline
  \end{tabular}
\caption{Corpora statistics for En$\rightarrow$De: number of parallel lines ($\times 10^3$) and proportion in the basic domain mixture. \domain{web} is the largest domain, containing about 54\% of the sentences, while \domain{bank} and \domain{tour} are very small.}
\label{tab:Corpora-en-de}
\end{table*}

We randomly select in each corpus a development and a test set of 1,000 lines each and keep the rest for training. Development sets help choose the best model according to the average BLEU score \cite{Papineni02bleu}.\footnote{We use truecasing and the \texttt{multibleu} script.}\fyDone{A word about meta-parameter settings}
   %    Significance testing is performed using bootstrap resampling \cite{Koehn04statistical}, implemented in compare-mt\footnote{\url{https://github.com/neulab/compare-mt}} \cite{Neubig19compare-mt}. We report significant differences at the level of $p=0.05$.
\fyFuture{Is this part on significance testing still accurate ?}

\subsection{Baseline architectures \label{ssec:baseline}}
\fyDone{Write this - settings and parameters for Mixed and Full-FT}
% Our baselines are standard for multi-domain systems.
% \footnote{We however omit domain-specific systems trained only with the corresponding subset of the data, which are always inferior to the mix-domain strategy \cite{Britz17mixing}.}
Using Transformers \cite{Vaswani17attention} implemented in OpenNMT-tf\footnote{\url{https://github.com/OpenNMT/OpenNMT-tf}} \cite{Klein17opennmt}, we train the following baselines:
\begin{itemize}
\item a generic model trained on a concatenation of all corpora, denoted \system{Mixed};\fyDone{Or mixed nat ?}
\item a fine-tuned model \cite{Luong15stanford,Freitag16fast}, based on the \system{Mixed} system, further trained on each domain with early stopping when the dev BLEU stops increasing during 3 consecutive epochs.
%  We again contrast two versions: full fine-tuning (\system{FT-Full}), which update all the parameters of the initial generic model \system{Mixed}; and the variants of \cite{Bapna19simple} (\system{FT-Block}).
\end{itemize}

For all En$\rightarrow$Fr models, we set the embeddings size and the hidden layers size to~512. Transformers use multi-head attention with 8 heads in each of the 6 layers; the inner feedforward layer contains 2,048 cells. Residual adapters additionally use an adaptation block in each layer, composed of a 2-layer perceptron, with an inner ReLU activation function operating on normalized entries of dimension $b=1024$.
% The gated variant is made of a dense layer, followed by a layer normalization and a sigmoid activation.
% The domain control systems are exactly as their baseline counterparts (RNN and Transformer), with an additional 2 cells encoding the domain on the input layer.
Training use a batch size of~12,288 tokens; optimization uses Adam with parameters $\beta_1=0.9$, $\beta_2= 0.98$ and Noam decay ($warmup\_steps=4,000$), and a dropout rate of $0.1$ for all layers.\fyDone{Describe the block adaptation layer - voir slides}

Models for En$\rightarrow$De are larger and rely on embeddings size as well as hidden layers size of~1024; each Transformers layer contains 16~attention heads; the inner feedforward layer contains 4,096 cells. Adapter modules have the same architecture as for the other language pair, except for their size, which is doubled ($b=2,048$). 
%The number of parameters in each model is reported in Table~\ref{tab:params}.

\iffalse{
\begin{table*}[htbp]
  \centering
  \begin{tabular}{|l|l|} \hline
    Model & params \\ \hline 
    \system{Transformer-En-Fr}  & 65 \\
    \system{Residual Adapter-En-Fr} & 1 \\
    \system{Transformer-En-De}  & 213 \\
    \system{Residual Adapter-En-De} & 8 \\
     \hline
  \end{tabular}
  \caption{Number of parameters ($\times 10^6$)}
  \label{tab:params}
\end{table*}
}
\fi

\subsection{Multi-domain systems}
In this section, we evaluate several proposals from the literature on multi-domain adaptation and compare them to full fine-tuning on the one hand, and to two variants of the residual adapter architecture on the other hand.
The reference methods included in our experiments are the following:
\begin{itemize}
\item a system using ``domain control'' \cite{Kobus17domaincontrol}. In this approach, domain information is introduced either as an additional token for each source sentence (\system{DC-Tag}) or in the form of a supplementary feature for each word (\system{DC-Feat});
\item a system using lexicalized domain representations \cite{Pham19generic}: word embeddings are composed of a generic and a domain-specific part (\system{LDR});
\item the three proposals of \newcite{Britz17mixing}. \system{TTM} is a feature-based approach where the domain tag is introduced as an extra word \textsl{on the target side}. The training uses reference tags and inference is performed with predicted tags, just like for regular target words. \system{DM} is a multi-task learner where a domain classifier is trained on top of the MT encoder, so as to make it aware of domain differences; \system{ADM} is the adversarial version of \system{DM}, pushing the encoder towards learning domain-independent source representations. These methods only use domain labels in training.
\end{itemize}

\begin{table*}[t!]
  \centering
  \begin{tabular}{|p{3cm}|*{8}{r|}} \hline
%     &&&&&& \\
    Model / Domain & \multicolumn{1}{c|}{\domain{ med}} & \multicolumn{1}{c|}{\domain{ law}} & \multicolumn{1}{c|}{\domain{bank}} & \multicolumn{1}{c|}{\domain{talk}} & \multicolumn{1}{c|}{\domain{ it }} & \multicolumn{1}{c|}{\domain{ rel}} & \multicolumn{1}{c|}{\domain{avg}} \\ \hline 
    \system{Mixed}        & 37.3 & 54.6 & 50.1 & 33.5 & 43.2 & 77.5  &  49.4 \\
    \system{FT-Full}       & 37.7 & 59.2 & 54.5 & 34.0 & 46.8 & 90.8 & 53.8 \\
    \hline 
    \system{DC-Tag}      & 38.1 & 55.3 & 49.9   & 33.2 & 43.5 & 80.5  & 50.1 \\
    \system{DC-Feat}     & 37.7 & 54.9 & 49.5   & 32.9 & 43.6 & 79.9 & 49.9  \\
    \system{LDR}            & 37.0  & 54.7 & 49.9 & 33.9 & 43.6 & 79.9 & 49.8    \\
    \system{TTM}           & 37.3  & 54.9 & 49.5 & 32.9 & 43.6 & 79.9 & 49.7   \\
    \system{DM}            & 35.6  & 49.5  & 45.6 & 29.9 & 37.1 & 62.4 & 43.4   \\ 
    \system{ADM}          & 36.4  & 53.5  & 48.3 & 32.0 & 41.5 & 73.4 & 47.5   \\
    \hline
    \system{Res-Adap}         & 37.3 & 57.9 & 53.9 & 33.8 & 46.7 & 90.2 & 53.3 \\ 
    \system{Res-Adap-MT}  & 37.9 & 56.0 & 51.2  & 33.5 & 44.4 & 88.3 & 51.9 \\
    %  \hfill MDL Res (gen)    & 37.7 & 51.0 & 34.0 & 30.4 & 34.2 & 15.2 & 36.4 & 33.7\\
    \hline
  \end{tabular}
  \caption{Baseline translation performance of various multi-domain MT systems (En$\rightarrow$Fr).}
  \label{tab:performance-multi}
\end{table*}
The two variants of the residual adapter model included in this first round of experiment have been presented in Section~\ref{ssec:architecture}: \system{Res-Adap} is the multi-domain version of the approach of \citet{Bapna19simple} based on a two step training procedure; while \system{Res-Adap-MT} is the ``multi-task'' version, where the parameters of the generic model and of the adapters are jointly learned from scratch.

Because of the limit of our computational resources, we restrict the experiments in this section to the En$\rightarrow$Fr task. Results are in Table~\ref{tab:performance-multi}.\fyFuture{Perform experiments for de:en}\fyFuture{Restore results where residual are removed from other paper.}

These results first show that full fine-tuning outperforms all other methods for the in-domain test sets. However, \system{Res-Adap} is able to reduce the gap with this approach for several domains, showing the effectiveness of residual adapters. The ``multi-task'' variant is slightly less effective in our experiments than the basic version, where optimization is performed in two steps. As it turns out, using residual adapters proves here more effective on average than the other reference multi-domain systems; it is also much better than the generic system for translating data from known domains, outperforming the \system{Mixed} system by more than 4 BLEU points in average. Gains are especially large for small domains such as \domain{law} and \domain{rel}.

% Multi-task training system \system{MDL-Res} gains important improvement compared to baseline \system{Mixed}. However, \system{MDL-Res} is outperformed by \system{Res-Adap}. It means that independently fine tuning the generic models is better than multi-task training. However, \system{MDL-Res} is better than other methods in multi-domain learning including:  \system{DC-Tag},  \system{DC-Feat}, \system{LDR}, \system{TTM}, \system{DM}, \system{ADM}.

\subsection{Varying the positions and number of residual adapters}
Tables~\ref{tab:performance-en-fr-pos-reg} and \ref{tab:performance-en-de-pos-reg} report the BLEU scores for 6 domains in each language pair: \domain{med},\domain{law},\domain{bank},\domain{talk},\domain{it} and \domain{rel} for En$\rightarrow$Fr; \domain{gov}, \domain{eco}, \domain{tour}, \domain{bank}, \domain{med} and \domain{news} for En$\rightarrow$De. We first see that for the latter direction, the basic version \system{Res-Adap} also outperforms the \system{mixed} baseline on average, with large gains for the small domains \domain{tour}, \domain{bank} and comparable results for the other domains.
% in almost all domains while stays equivalent to the baseline in \domain{med} in En$\rightarrow$Fr and \domain{news}, \domain{gov}, \domain{eco}, \domain{med} in En$\rightarrow$De.
   %    However, \system{FT-full} still remains the best model for all test domains.
\fyFuture{Full FT for en->DE ? }

By varying the number and position of residual adapters (see Section~\ref{ssec:architecture}), we then contrast several implementation of residual adapters. \fyDone{Fix style here} Because the set of possible configurations is large, we only perform experiments for layers $i= 2, 4, 6$ (both for the encoder and decoder). Two settings are considered: keeping just one adapter, or keeping the three. The trend is the same for the two language directions: suppressing adapters always hurts the overall performance, albeit by a small margin: having six adapters is better than three, which is better than keeping only one. With only one adapter active, we observe small, unsignificant changes in performance when varying the adapter's depth. Based on these experiments, it seems that adaptation remains useful even for the deeper layers of the architecture.\fyFuture{More configurations ?}

\begin{table*}[htbp]
  \centering
  \begin{tabular}{|p{3cm}|*{8}{r|}} \hline
%     &&&&&& \\
    Model / Domain & \multicolumn{1}{c|}{\domain{ med}} & \multicolumn{1}{c|}{\domain{ law}} & \multicolumn{1}{c|}{\domain{bank}} & \multicolumn{1}{c|}{\domain{talk}} & \multicolumn{1}{c|}{\domain{ it }} & \multicolumn{1}{c|}{\domain{ rel}} & \multicolumn{1}{c|}{\domain{avg}} & \multicolumn{1}{c|}{\domain{params}} \\ \hline 
    \system{Mixed}  & 37.3 & 54.6 & 50.1 & 33.5 & 43.2 & 77.5  & 49.4 & 65M/0 \\
    \system{Res-Adap}     & 37.3 & 57.9 & 53.9 & 33.8 & 46.7 & 90.2 & 53.3 & 65M/12M\\ \hline
    \system{Res-Adap$_{(2,4,6)}$}     & 37.7 & 57 & 53 & 33.3 & 45 & 90 & 52.7 & 65M/6M\\
    \system{Res-Adap$_{(6)}$}     & 37.7 & 55.8 & 51.5 & 33.9 & 43.6 & 89.2 & 51.9 & 65M/2M \\
    \system{Res-Adap$_{(4)}$}     & 37.9 & 55.6 & 51.7 & 33.7 & 44.4 & 88.7 & 52 & 65M/2M\\
    \system{Res-Adap$_{(2)}$}     & 37.8 & 55.5 & 51.4 & 34 & 43.8 & 86.7 & 51.5 & 65M/2M\\ \hline
    \system{Res-Adap-WD}     & 37.2 & 56.0 & 52.9 & 33.4 & 46.0 & 90.6 & 52.7 & 65M/12M \\
    \system{Res-Adap-LR}      & 37.4 & 56.1 & 51.8 & 33.3 & 45.0 & 89.7 & 52.2 & 65M/12M \\  
     \hline
  \end{tabular}
  \caption{Translation performance of various fine-tuned systems (En$\rightarrow$Fr). We report BLEU scores for each domain, as well as averages across languages. Column \domain{params} reports the number of domain-agnostic/domain-specific parameters \label{tab:performance-en-fr-pos-reg}} \fyDone{Boldface ?}
\end{table*}

\begin{table*}[htbp]
  \centering
  \fyDone{Fix column size}
  \begin{tabular}{|p{3cm}|*{8}{r|}} \hline
%     &&&&&& \\
    Model / Domain & \multicolumn{1}{c|}{\domain{gov}} & \multicolumn{1}{c|}{\domain{eco}} & \multicolumn{1}{c|}{\domain{tour}} & \multicolumn{1}{c|}{\domain{bank}} & \multicolumn{1}{c|}{\domain{ med }} & \multicolumn{1}{c|}{\domain{ news}} & \multicolumn{1}{c|}{\domain{avg}} & \multicolumn{1}{c|}{\domain{params}} \\ \hline
    \system{Mixed}          & 29.3 & 30.5 & 17.6 & 38.1 & 47.9 & 20.9  & 30.6 & 213M/0M\\
   \system{Res-Adap}     & 29.6 & 30.4 & 19.2 & 49.0 & 47.2 & 20.6 & 33.1 & 213M/96M \\ \hline
    \system{Res-Adap$_{(2,4,6)}$} & 29.7  & 30.5 & 18.8 & 49.6 & 47.1 & 20.6 &  32.7 & 213M/48M \\ \hline
    \system{Res-Adap$_{(6)}$}      & 29.5 & 30.4 & 18.1 & 49.1 & 46.9 & 20.4 & 32.4 & 213M/16M \\
   \system{Res-Adap$_{(4)}$}       & 29.7 & 30.4 & 18.1 & 49.6 & 47.0 & 20.6 & 32.6 & 213/16M\\
   \system{Res-Adap$_{(2)}$}       & 29.6 & 30.4 & 18.3 & 49.4 & 46.7 & 20.6 & 32.5  & 213M/16M\\
    \system{Res-Adap-WD}         & 29.7 & 30.8 & 20.4 & 50.2 & 47.7 & 20.6 & 33.2 & 213M/96M \\
    \system{Res-Adap-LR}           & 29.6 & 30.4 & 19.2 & 49.0 & 47.2 & 20.6 & 33.1  & 213M/96M\\
    \hline
  \end{tabular}
  \caption{Translation performance of various fine-tuned systems (En$\rightarrow$De). We report BLEU scores for each domain, as well as averages across domains. In column \domain{params}, we report the number of domain-agnostic/domain-specific parameters}
  \label{tab:performance-en-de-pos-reg}
\end{table*}

\subsection{Regularizing fine-tuning \label{ssec:regularization-exp}}

The translation from English into German includes two domains (\domain{tour} and \domain{bank}) that are extremely small and account only for a very small fraction of the training data (respectively for 0.039\% and 0.022\% of the total number of sentences). Fine-tuning on these domains can lead to serious overfitting. We assess two well-known regularization techniques for adapter modules, that could help mitigate this problem: weight decay and layer regularization. 

For each method, the optimal hyper-parameters (weight decay or layer regularization coefficient, see Section~\ref{sssec:design-space}) $\lambda$ by chosen by grid search among a small set of values ($\{ 10^{-3}, 10^{-4}, 10^{-5} \}$).\fyFuture{Better grid even better EWC.}

Results in Tables~\ref{tab:performance-en-fr-pos-reg} and \ref{tab:performance-en-de-pos-reg} show that regularizing the adapter model can positively affect the test performance for the smallest domains (this is especially clear for weight-decay (\system{Res-Adap-WD}) in En$\rightarrow$De), at the cost however of a small drop in performance for the other domains. Using layer regularization proves here to be comparatively less effective. Finding better ways to set the regularization parameters, for instance by varying $\lambda$ for each domain based on the available supervision data, is left for future work.  
\fyDone{How is the weight decay parameter set ?}
\fyDone{Why is regularization not helping ? It helps for small domain - domain-specific regularization ??}

\subsection{Highway and Gated Residual Adaptaters \label{ssec:gate-exp}}

We now turn to the evaluation of our new architectural variants: Highway residual adapters \system{Res-Adap-HW} on the one hand, and Gated residual adapters \system{Res-Adap-Gated} on the other hand. We use the same domains and settings as before, focusing here exclusively on the language direction En$\rightarrow$Fr.

To also evaluate the robustness with respect to out-of-domain examples, we perform two additional experiments. We first generate translations with erroneous (more precisely: randomly assigned) domain information: the corresponding results appear in Table~\ref{tab:performance-random} under the column \domain{rnd}. We also compute translation for a domain unseen in training (\domain{news}) as follows. For each sentence of this test set, we automatically evaluate the closest domain,\footnote{As measured by the perplexity of a language model trained with only in-domain data.\fyFuture{More on this for reproducibility}.} then use the predicted domain label to compute the translation. This is an error prone process, which also challenges the robustness of our multi-domain systems. Results are in Table~\ref{tab:performance-random}.
\fyFuture{This is a baseline, and we have not discussed it - keep for next time: We also report model \system{FT-Full-UEW} which is obtained by full-finetuning the generic model using uniform elastic weight to avoid catastrophic forgetting.}

A first observation is that for domains seen in training, our variants \system{Res-Adap-HW} and \system{Res-Adap-Gated} achieve BLEU scores that are on a par to those of the original version (\system{Res-Adap}), with insignificant variations across test sets. \fyFuture{Comment on processing time.}

The two other settings are instructive in several ways: they first clearly illustrate that the brittleness of domain-adapted systems, for which large drops in performance (more than 15 BLEU points on average) are observed when the domain label is randomly chosen. Our gated variant however proves much more robust than the other adaptation strategy, and performs almost on par to the generic system for that test condition. The same trend holds for the unseen \domain{news} domain, with \system{Res-Adap-Gated} being the best domain adapted system in our set, outperforming the other variants by about 2 BLEU points.
% The phenomenon of Catastrophic Forgetting manifests clearly in this situation where the performance of all multi-domain systems drops dramatically by approximately 10 points BLEU compared to generic model \system{Mixed}. Our proposed model \system{FT-Gate-Block} maintains equivalent performance to the generic model in \system{RND} thanks to the application of adaptive weight on the output of the residual adapter. \system{FT-Gate-Block} avoids relying on the residual adapters when predicting examples that come from domain far from the domain of the adapters. 

% In the new domain test \domain{news}, generic model outperforms other systems. Finetuning systems including \system{FT-Full}, \system{Res-Adap} and \system{FT-HW-Block} underperforms generic model by large margin. More regularized variants including \system{FT-Full-UEW}, and \system{FT-Gate-Block} are able to reduce the gap of performance compared to generic.

% The column \domain{PROC} shows that highway residual adapter \system{FT-HW-Block} process examples faster than the original version \system{Res-Adap}.

% \fyTodo{Why HW worse than standard version ?} 

\begin{table*}[htbp]
  \centering
  \begin{tabular}{|p{3cm}|*{7}{r|}|r|r|} \hline
%     &&&&&& \\
    Model / Domain & \multicolumn{1}{c|}{\domain{ med}} & \multicolumn{1}{c|}{\domain{ law}} & \multicolumn{1}{c|}{\domain{bank}} & \multicolumn{1}{c|}{\domain{talk}} & \multicolumn{1}{c|}{\domain{ it }} & \multicolumn{1}{c|}{\domain{ rel}} & \multicolumn{1}{c||}{\domain{avg}} & \multicolumn{1}{c|}{\domain{rnd}} & \multicolumn{1}{c|}{\domain{news}} \\ \hline % & \multicolumn{1}{c|}{\domain{PROC}} \\ \hline  
    \system{Mixed}             & 37.3 & 54.6 & 50.1 & 33.5 & 43.2 & 77.5     &  49.4 & 49.4 & 23.5 \\ %& 34s \\ %
    \system{FT-Full}             & 37.7 & 59.2 & 54.5 & 34.0 & 46.8 & 90.8   & 53.8 & 32.5 & 20.2  \\ %& 34s \\
    \system{Res-Adap}         & 37.3 & 57.9 & 53.9 & 33.8 & 46.7 & 90.2   & 53.3 & 38.4 & 20.5 \\ %& 22s\\ 
    \system{Res-Adap-HW}   & 37.5 & 57.2 & 53.4 & 33.1 & 46.3 & 91.0  & 53.1 & 36.6 & 20.2 \\ %& 19s \\
    \system{Res-Adap-Gate}  & 38.0 & 57.5& 53.0 & 33.5 & 46.0 & 90.1  & 53.0 & 49.0 & 22.5 \\ %& 21s \\
%    \system{FT-Full-UEW}      & 37.9 & 56.0 & 52.1 & 33.7 & 44.9 & 89.1 &	52.3 & 46.5 & 22.11 \\ %& 34s \\
    \hline
  \end{tabular}
  \caption{Translation performance of highway and gated variants for En$\rightarrow$Fr.
    % Column \domain{PROC} is time (in seconds) of 100 iterations with batch size 12,288 tokens
  }
  \label{tab:performance-random}
\end{table*}
\fyFuture{Computation time should also report the time it takes for the complete tuning process, not one iteration which will be more or less the same}.
\section{Related Work \label{sec:related}}
\mpDone{related work}

Training with data from multiple, heterogeneous sources is a common scenario in natural language processing \cite{Dredze08online,Finkel09hierarchical}. It is thus no wonder that the design of multi-domain systems has been proposed for many tasks. In this short survey, we exclusively focus on machine translation; it is likely that similar methods (parameter sharing, instance selection / weighting, adversarial training, etc) have also been proposed for other tasks.

Early approaches to multi-domain MT were proposed for statistical MT, either considering multiple data sources (eg.\ \cite{Banerjee10combining,Clark12onesystem,Sennrich13multidomain,Huck15mixeddomain}) or domains containing several topics \cite{Eidelman12topic,Hasler14dynamic-topic}. Two main strategies emerge: feature-based methods, where domain labels are integrated through supplementary features; and instance-based methods, involving a measure of similarity between train and test domains. 

The former approach has also been adapted to NMT: \newcite{Kobus17domaincontrol,Tars18multidomain} use an additional domain feature in an RNN model, in the form of an extra domain-token or of additional domain-features associated with each word. \citet{Chen16guided} apply domain control on the \emph{target} side, using a topic vector to describe the whole document context. Similar ideas are developed in \cite{Chu18multilingual,Pham19generic}, where domain differences and similarities are enforced through parameter sharing schemes. Parameter-sharing also lies at the core of the work by \citet{Jiang19multidomain}, who consider a Transformer model containing both domain-specific and domain-agnostic heads.

\citet{Britz17mixing} study three general techniques to take domain information into account in training: they rely on either domain classification or domain normalization on the source or target side. A contribution of this study is an adversarial training scheme to normalize representations across domains and make the combination of multiple data sources more effective. Similar techniques (parameter sharing, automatic domain classification / normalization) are at play in \cite{Zeng18multidomain}: in this work, the lower layers of the MT use auxiliary classification tasks to disentangle domain specific from domain-agnostic representations. These representations are first processed separately, then merged to compute the final translation.

\citet{Farajian17multidomain,li-etal-2018-one} are two recent representatives of the instance-based approach: for each test sentence, a small adaptation corpus is collected based on similarity measures, and used to fine-tune a mix-domain model.

Finally note that a distinct evolution of the residual adapter model of \cite{Bapna19simple} is presented in \cite{Sharaf20metalearning} where meta-learning techniques are used to make fine tuning more effective in a standard domain-adaptation setting.

\section{Conclusion and outlook \label{sec:discussion}}
\mpDone{discussion}
In this paper, we have performed an experimental study of the residual adapter architecture in the context of multi-domain adaptation, where the goal is to build one single systems that (a) performs well for domain seen in training, ideally as well as full fine-tuning; (b) is also able to robustly handle translations for new, unseen domains. We have shown that this architecture allowed us to quickly adapt a model to a specific domain, delivering BLEU performance than are much better than the generic, mixed domain baseline, and close the gap with the full-finetuning approach, at a modest computational cost. Several new variants have been introduced and evaluated for two language directions: if none that able to clearly surpass the baseline residual adapter models, they provide directions for improving this model in practical settings: unbalanced data condition, noise in label domains, etc. In our future work, we would like to continue the development of the gated variant, which, it seems to us, provides a flexible and robust tool to address the various challenges of multi-domain machine translation. 

% without creating lots of domain-specific parameters even though full finetuning still outperforms this method by an important gap in in-domain test sets. For extremely small domains, we are able to reduce the number of adapters incorporated to the NMT model without losing performance, also using Weight-Decay in finetuning over small domains demonstrates significant improvement. The proposed highway variant shows an equivalent performance to the original version of \cite{Bapna19simple} which allows us to process the hidden layer of the NMT model only once. The extended version of the highway residual adapter to gated residual adapter shows the benefit in the robustness against wrong domain prediction and out-of-domain examples. 

\fyFuture{Example of gating ?}
\fyTodo{Why MDMT does not work well - full train from generic? - add new results}
\fyFuture{Structured domain? A Bayesian version?}
\fyFuture{Let the machine decide domains in learning? - domains and their organization - how could it?}

%\section*Acknowledgments}
\fyFuture{Thank you Jean Zay}

\bibliographystyle{acl_natbib}
\bibliography{mdadapt}
%\appendix
%\section{Appendices}
%\label{sec:appendix}
%\mpTodo{appendix}
%\section{Supplemental Material}
%\label{sec:supplemental}

\end{document}
