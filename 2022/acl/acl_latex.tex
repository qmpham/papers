% This must be in the first 5 lines to tell arXiv to use pdfLaTeX, which is strongly recommended.
\pdfoutput=1
% In particular, the hyperref package requires pdfLaTeX in order to break URLs across lines.

\documentclass[11pt]{article}

% Remove the "review" option to generate the final version.
\usepackage[review]{acl}

% Standard package includes
\usepackage{times}
\usepackage{latexsym}

% For proper rendering and hyphenation of words containing Latin characters (including in bib files)
\usepackage[T1]{fontenc}
% For Vietnamese characters
% \usepackage[T5]{fontenc}
% See https://www.latex-project.org/help/documentation/encguide.pdf for other character sets

% This assumes your files are encoded as UTF8
\usepackage[utf8]{inputenc}

% This is not strictly necessary, and may be commented out,
% but it will improve the layout of the manuscript,
% and will typically save some space.
\usepackage{microtype}

% If the title and author information does not fit in the area allocated, uncomment the following
%
%\setlength\titlebox{<dim>}
%
% and set <dim> to something 5cm or larger.

%algo packages
\usepackage{algorithm}
\usepackage{algorithmic}

%color packages
\usepackage{xcolor}
\colorlet{shadecolor}{orange!15}
\usepackage{color,soul}
\usepackage{soulutf8}
\definecolor{bordeau}{rgb}{0.3515625,0,0.234375}

% position of figure
\usepackage[absolute,overlay]{textpos}

%graphic packages
\usepackage{graphicx}
\graphicspath{ {figures/} }
\usepackage{wrapfig}
\usepackage{float} 
\usepackage{tikz}
\usepackage{pgfplots}
\usepackage{pgfplotstable}

% table packages
\usepackage{array}
\usepackage{multirow,booktabs}
\usepackage{multicol}
\usepackage{tabularx}

% font package
\usepackage[utf8]{inputenc}	% Para caracteres en español
\usepackage{textcomp}
%\usepackage{times}
%\usepackage{lmodern}
\usepackage{mathptmx}

% equation packages
\usepackage{amsmath,amsthm,amsfonts,amssymb,amscd}
\usepackage{mathrsfs}

\title{Frustratingly easy multi-domain, multilingual translation with latent multi-task group dropout}

% Author information can be set in various styles:
% For several authors from the same institution:
% \author{Author 1 \and ... \and Author n \\
%         Address line \\ ... \\ Address line}
% if the names do not fit well on one line use
%         Author 1 \\ {\bf Author 2} \\ ... \\ {\bf Author n} \\
% For authors from different institutions:
% \author{Author 1 \\ Address line \\  ... \\ Address line
%         \And  ... \And
%         Author n \\ Address line \\ ... \\ Address line}
% To start a seperate ``row'' of authors use \AND, as in
% \author{Author 1 \\ Address line \\  ... \\ Address line
%         \AND
%         Author 2 \\ Address line \\ ... \\ Address line \And
%         Author 3 \\ Address line \\ ... \\ Address line}

\author{First Author \\
  Affiliation / Address line 1 \\
  Affiliation / Address line 2 \\
  Affiliation / Address line 3 \\
  \texttt{email@domain} \\\And
  Second Author \\
  Affiliation / Address line 1 \\
  Affiliation / Address line 2 \\
  Affiliation / Address line 3 \\
  \texttt{email@domain} \\}

\begin{document}
\maketitle
\begin{abstract}
Multi-domain and multilingual translation can be cast under multi-task learning in which each domain, each language pair are considered as a task. Sharing most of the parameters of the model causes a catastrophic interference between unrelated tasks. In this work, we propose equivalently grouping the nodes of each layer and dropping a portion of these groups assigning different sparse sub-networks to different task. We demonstrate a significant improvement in multi-domain and multilingual translation. Furthermore, through an ablation study, we show that the learned dropping patterns are similar in related tasks allowing the knowledge transferring between these task.
\end{abstract}

\section{Introduction}
Multi-domain translation and multilingual translation develop one model for translations in multi-domain and multiple language pairs respectively. These paradigms are motivated by the compactness of the translation model and the hypothetical positive knowledge transfer between similar domains \citep{Pham21revisiting} and between languages in the same family \citep{Tan19multilingual}. 

The method we developed here mitigate the catastrophic interference occurring in training a neural machine translation (NMT) model for multiple domains or multiple language pairs. 

\section{Multi-task group dropout}
We define a group dropout for a domain $d$ as follow
\begin{equation}
\begin{array}{rcl}
\tilde{h}^l &=& h^l \odot r_l^d ,\\
h^{l+1} &=& \text{LAYER}^{l+1}(h^l) ,\\
l & \in & [0,\cdots,L] ,\\
\end{array}
\end{equation}
where $L$ is number of Transformer layers of our NMT model.

The dropping mask $r_l^d$ of $l^{th}$ layer and domain $d$ is defined as follow
\begin{equation}
\begin{array}{rcl}
r_l^d(i) &=& \begin{cases}
      1, & \text{if}\ p \times \frac{d_k}{n_p} \leqslant i < (p+1) \times \frac{d_k}{n_p} \\
      & \text{AND} \  m_l^d(p) == 1 \\
      0, & \text{otherwise}
    \end{cases} \\
p & \in & \{0,\dots,n_p-1 \} \\
d & \in & \{0,\dots,n_d-1 \} \\
i & \in & \{0,\dots,d_k-1 \} \\
m_l^d & \in & \{0,1\}^{n_p} \ \forall d,l \\
\mid m_l^d \mid_{L_1} & = & k  \\
\Delta^{n_p}_k & = & \{ m \in \{0,1\}^{n_p} | \mid m \mid_{L_1} = k \}
\end{array}
\end{equation}
where $n_p$ is the number of dropout groups; $n_d$ is the number of domains; $d_k$ is the size of Transformer layers; $k$ is the number of retained groups at each layer; $m_l^d$ is a binary vector indicating which group of the nodes of the $l^{th}$ layer is dropped in the sub-network modeling domain $d$. The group $p$ is retained for domain $d$ if $m_l^d(p)==1$ and is dropped if $m_l^d(p)==0$. We also denote $\Delta^{n_p}_k$ the set of all admissible dropping masks. 

We incorporate the latent variables $m_{l\in[0,\cdots,L]}^d$ into the log-likelihood of our model as below. For the sake of brevity, we denote $m_l^d$ the set of latent variables $m_{l\in[0,\cdots,L]}^d$
\begin{equation}
\hspace{-1.5em}
\begin{array}{rcl}
\text{log} P(y|x,\theta,d) & \text{=} & \text{log} P(y, m_l^d |x,\theta,d) \text{-} \text{log} P(m_l^d|y, x,\theta,d) \\
& \text{=} & \mathbb{E}_{P(m_l^d | \Phi_l^d)} \big[ \text{log} P(y, m_l^d | x,\theta,d) \text{-} \text{log} P(m_l^d | \Phi_l^d) \big] \\
& \text{+} & KL(P(m_l^d | \Phi_l^d) \parallel P(m_l^d|y, x,\theta,d)) \\
& \geq & \mathbb{E}_{P(m_l^d | \Phi_l^d)} \big[ \text{log} P(y, m_l^d | x,\theta,d) \text{-} \text{log} P(m_l^d | \Phi_l^d) \big]  \\
& \text{=} & \mathbb{E}_{P(m_l^d | \Phi_l^d)} \big[ \text{log} P(y | x,\theta,m_l^d, d) \big] \\
& \text{-} & KL \big( P(m_l^d | \Phi_l^d) \parallel P(m_l^d | x,\theta,d) \big) \\
& \text{=} & \mathcal{L}(\theta, \Phi_l^d, x , y) \label{eq:lower-bound}
\end{array}
\end{equation}
our training objective will be the lowerbound, which is called the evidence lower bound (ELBO) in variational statistic \citep{Diederick14auto}, $\mathcal{L}(\theta, \Phi_l^d, x , y)$. The variational distribution of $m_l^d$, $P(m_l^d|\Phi_l^d)$, is defined as follows
\begin{align*}
\Phi_d & \in \mathbb{R}^{n_p} \\
Ind^d & \text{=} \{ i_1, \cdots, i_k \} \ | \ \Phi_l^d \\
& \sim \ \text{\small Sampling\_without\_replacement} \big(\text{softmax}(\Phi_l^d) \big) \\
m_l^d(i) &\text{=} \mathbb{I}(i \in Ind^d) \\
\end{align*}

The choice of the modeling of the latent vector $m_l^d$ is motivated by the Gumbel Top-K trick \citep{Kool19stochastic}. In addition, we suppose the prior distribution of $m_l^d$ is uniform over the admissible set $\Delta^{n_p}_k$, $P(m_l^d | x, \theta, d) \ \text{=} \ U(\Delta^{n_p}_k)$.

With our assumption over the prior distribution and our choice of the variational distribution, we express the two terms of the lowerbound \eqref{eq:lower-bound} as follows

\begin{equation}
\begin{array}{rcl}
T_1 &=& \mathbb{E}_{P(m_l^d | \Phi_l^d)} \big[ \text{log} P(y | x,\theta,m_l^d, d)  \big] \\
&=& \mathbb{E}_{g_l^d \sim Gumbel(0,1) \ \text{i.i.d}} \big[ \text{log} P(y | x,\theta,\tilde{m}_l^d,d) \big] \\
\end{array}
\end{equation}
where
\begin{equation}
\begin{array}{rcl}
g_l^d & \in & \mathbb{R}^{n_p} \\
g_l^d(p) & \sim & Gumbel(0,1) \ \text{i.i.d} \\
&& \forall p \in \{ 0,\cdots,n_p-1 \} \\
i_1, \cdots, i_k &=& Top_{k} \{ g_l^d(0) + \Phi_l^d(0), \cdots,\\ &&g_l^d(n_d-1) + \Phi_l^d(n_d-1) \} \\
\tilde{m}_l^d(p) &=& \mathbb{I}(p \in Ind^d). \\
\end{array}
\end{equation}

\begin{equation}
\begin{array}{rcl}
T_2 &=& - KL \big( P(m_l^d | \Phi_l^d) \parallel P(m_l^d | x,\theta,d) \big) \\
	&=& \mathbb{H} \big[ P(m_l^d | \Phi_l^d) \big] - \text{log}(\#\Delta^{n_p}_k) \\ \label{eq:entropy}
\end{array}
\end{equation}
we will prove this inequality in Appendix \ref{appendix:b}.
Our loss function, which is the negative of the lowerbound \eqref{eq:lower-bound}, consists of 2 terms $-T_1$ and $-T_2$. The inequality \eqref{eq:entropy} show that the upperbound of $-T_2$ is the entropy $\mathbb{H}(\text{softmax}(\Phi_l^d))$ with 

%\begin{equation}
%\begin{array}{rcl}
%L(\theta, \Phi_0, \cdots, \Phi_{n_d\text{-1}}) &=& \sum_{d=0}^{n_d-1} \mathbb{E}_{x \sim \mathcal{D}_s^d, y \sim g^d(x)} \big[ \mathcal{L}(\theta,\Phi_d,x,y) \big]
%\end{array}
%\end{equation}


%
%\begin{equation}
%\begin{array}{rcl}
% \tilde{m}^d &=&\displaystyle{\mathop{argmin}_{\substack{
%       0 \leq m_i \leq 1 \ \forall 0 \leq i \leq n_d\text{-}1 \\
%        1^{T}.m = k
%      }}} -(g+\Phi_l^d)^{T} . m - \tau H_b(m) \\
%& & \\
%H_b(m) &=& - \sum_i m_i \text{log}(m_i) + (1-m_i)\text{log}(1-m_i)
%\end{array}
%\end{equation}
%
%\begin{equation}
%\begin{array}{rcl} 
%
%Ind^d &=& Top_k(\Phi_l^d) \\
%m_l^d &=& \mathbb{I}(i\in Ind^d)\\
%
%y &\sim& P(.|x,\theta,m_l^d)
%
%\end{array}
%\end{equation}
\section{Experimental settings}
\section{Results and ablation studies}
\section{Related work}
\section{Conclusions and outlook}
\bibliography{bibliography}
\bibliographystyle{acl_natbib}

\appendix
\section{Appendix A}
\label{appendix:a}
Theorem of implicit differentiation

\section{Appendix B}
\label{appendix:b}
Entropy inequality
\section{Appendix C}
\label{appendix:c}
Data size


%\label{sec:appendix}

\end{document}
